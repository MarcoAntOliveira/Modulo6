
    \documentclass{article}
    \usepackage[legalpaper, left=1 cm, right=1cm, top=0.5cm, bottom=0.5cm] {geometry}
    \date{} % Remove a exibição da data
    \usepackage{xcolor}
    \usepackage{listings}
    \usepackage{graphicx}
    \usepackage{hyperref} % Para criar links
    \usepackage[utf8]{inputenc}
    \usepackage[T1]{fontenc}

    \lstset{
    language=python,
    basicstyle=\ttfamily,
    keywordstyle=\bfseries\color{blue},
    commentstyle=\color{blue},
    stringstyle=\color{red!70!black},
    numberstyle=\tiny,
    stepnumber=1,
    numbersep=5pt,
    backgroundcolor=\color{white},
    breaklines=true,
    breakautoindent=true,
    showspaces=false,
    showstringspaces=false,
    showtabs=false,
    tabsize=2,
    literate={~}{{\textasciitilde}}1, % Trata ~ como um caractere normal
    extendedchars=true, % Permite caracteres estendidos (acentos, etc.)
    inputencoding=utf8, % Define a codificação de entrada como UTF-8
    literate={á}{{\'a}}1 {ã}{{\~a}}1 {ç}{{\c{c}}}1
    }
    \title{Modulo 6}
    \begin{document}
    \maketitle
    \tableofcontents
    \section{Criando meu primeiro arquivo do SQLite (db.sqlite3)}
    \begin{lstlisting}
        import sqlite3
from pathlib import Path

ROOT_DIR = Path(__file__).parent
DB_NAME = 'db.sqlite3'
DB_FILE = ROOT_DIR / DB_NAME

connection = sqlite3.connect(DB_FILE)
cursor = connection.cursor()

# SQL

cursor.close()
connection.close()
    \end{lstlisting}

    \section{Criando minha primeira tabela no SQLite3 (DBeaver opcional)}
    \begin{lstlisting}
    ROOT_DIR = Path(__file__).parent
DB_NAME = 'db.sqlite3'
DB_FILE = ROOT_DIR / DB_NAME
TABLE_NAME = 'customers'

connection = sqlite3.connect(DB_FILE)
cursor = connection.cursor()

# SQL
cursor.execute(
    f'CREATE TABLE IF NOT EXISTS {TABLE_NAME}'
    '('
    'id INTEGER PRIMARY KEY AUTOINCREMENT,'
    'name TEXT,'
    'weight REAL'
    ')'
)
connection.commit()

cursor.close()
connection.close()
    \end{lstlisting}

    \section{inserindo valores (INSERT INTO), DELETE sem WHERE e zerando a sqlite}
    \begin{lstlisting}
        connection = sqlite3.connect(DB_FILE)
cursor = connection.cursor()

# CUIDADO: fazendo delete sem where
cursor.execute(
    f'DELETE FROM {TABLE_NAME}'
)
cursor.execute(
    f'DELETE FROM sqlite_sequence WHERE name="{TABLE_NAME}"'
)
connection.commit()

# Cria a tabela
cursor.execute(
    f'CREATE TABLE IF NOT EXISTS {TABLE_NAME}'
    '('
)
connection.commit()

# Registrar valores nas colunas da tabela
# CUIDADO: sql injection
cursor.execute(
    f'INSERT INTO {TABLE_NAME} '
    '(id, name, weight) '
    'VALUES '
    '(NULL, "Helena", 4), (NULL, "Eduardo", 10)'
)
connection.commit()

cursor.close()
connection.close()
    \end{lstlisting}

    \section{Usando placeholders}
    \textbf{Esse codigo é uma continuação do codigo anterior}
    \begin{lstlisting}
        connection.commit()

# Registrar valores nas colunas da tabela
# CUIDADO: sql injection
cursor.execute(
sql = (
    f'INSERT INTO {TABLE_NAME} '
    '(id, name, weight) '
    '(name, weight) '
    'VALUES '
    '(NULL, "Helena", 4), (NULL, "Eduardo", 10)'
    '(?, ?)'
)
cursor.execute(sql, ['Joana', 4])
connection.commit()
print(sql)

cursor.close()
connection.close()
    \end{lstlisting}

    \section{Inserindo vários valores com execute many}
    \begin{lstlisting}
        'VALUES '
        '(?, ?)'
    )
    cursor.execute(sql, ['Joana', 4])
    # cursor.execute(sql, ['Joana', 4])
    cursor.executemany(
        sql,
        (
            ('Joana', 4), ('Luiz', 5)
        )
    )
    connection.commit()
    print(sql)
    \end{lstlisting}
    \section{execute e executemany com dicionários e lista de dicionários}
    \begin{lstlisting}
        f'INSERT INTO {TABLE_NAME} '
    '(name, weight) '
    'VALUES '
    '(?, ?)'
    '(:nome, :peso)'
)
# cursor.execute(sql, ['Joana', 4])
cursor.executemany(
    sql,
    (
        ('Joana', 4), ('Luiz', 5)
    )
)
# cursor.executemany(
#     sql,
#     (
#         ('Joana', 4), ('Luiz', 5)
#     )
# )
cursor.execute(sql, {'nome': 'Sem nome', 'peso': 3})
cursor.executemany(sql, (
    {'nome': 'Joãozinho', 'peso': 3},
    {'nome': 'Maria', 'peso': 2},
    {'nome': 'Helena', 'peso': 4},
    {'nome': 'Joana', 'peso': 5},
))
connection.commit()
print(sql)


    \end{lstlisting}

    \section{SELECT do SQL com fetch no SQLite3 do Python}
    \textbf{main.py}
    \begin{lstlisting}
        {'nome': 'Joana', 'peso': 5},
        ))
        connection.commit()
        print(sql)
        
        cursor.close()
        connection.close()
        
        if __name__ == '__main__':
            print(sql)
    \end{lstlisting}

    \textbf{select.py}
    \begin{lstlisting}
        import sqlite3

from main import DB_FILE, TABLE_NAME

connection = sqlite3.connect(DB_FILE)
cursor = connection.cursor()

cursor.execute(
    f'SELECT * FROM {TABLE_NAME}'
)

for row in cursor.fetchall():
    _id, name, weight = row
    print(_id, name, weight)


print()

cursor.execute(
    f'SELECT * FROM {TABLE_NAME} '
    'WHERE id = "3"'
)
row = cursor.fetchone()
_id, name, weight = row
print(_id, name, weight)

cursor.close()
connection.close()
    \end{lstlisting}

    \section{O que é CRUD + DELETE com e sem WHERE no SQLite3 do Python}
    
    \begin{lstlisting}
        connection = sqlite3.connect(DB_FILE)
cursor = connection.cursor()

# CRUD - Create Read   Update Delete
# SQL -  INSERT SELECT UPDATE DELETE

# CUIDADO: fazendo delete sem where
cursor.execute(
    f'DELETE FROM {TABLE_NAME}'
)

# DELETE mais cuidadoso
cursor.execute(
    f'DELETE FROM sqlite_sequence WHERE name="{TABLE_NAME}"'
)
@@ -51,7 +56,6 @@
    {'nome': 'Joana', 'peso': 5},
))
connection.commit()

cursor.close()
connection.close()

    \end{lstlisting}
    
    \section{DELETE no SQLite do Python}
    \begin{lstlisting}
        {'nome': 'Joana', 'peso': 5},
        ))
        connection.commit()
        cursor.close()
        connection.close()
        
        
        if __name__ == '__main__':
            print(sql)
        
            cursor.execute(
                f'DELETE FROM {TABLE_NAME} '
                'WHERE id = "3"'
            )
            cursor.execute(
                f'DELETE FROM {TABLE_NAME} '
                'WHERE id = 1'
            )
            connection.commit()
        
            cursor.execute(
                f'SELECT * FROM {TABLE_NAME}'
            )
        
            for row in cursor.fetchall():
                _id, name, weight = row
                print(_id, name, weight)
        
            cursor.close()
            connection.close()
    \end{lstlisting}

    \section{UPDATE no SQLite com Python}
        \begin{lstlisting}
                )
            connection.commit()

            cursor.execute(
                f'UPDATE {TABLE_NAME} '
                'SET name="QUALQUER", weight=67.89 '
                'WHERE id = 2'
            )
            connection.commit()

            cursor.execute(
                f'SELECT * FROM {TABLE_NAME}'
            )

        \end{lstlisting}
        \section{Pymysql}
    
        \textbf{main.py}
        \begin{lstlisting}
            import pymysql
            import dotenv #type:ignore
            import os
            dotenv.load_dotenv()

            connection = pymysql.connect(
                host =os.environ['MYSQL_HOST'],
                user = os.environ['MYSQL_USER'],
                password =os.environ['MYSQL_PASSWORD'],
                database =os.environ['MYSQL_DATABASE']
            )

            print(os.environ['MYSQL_HOST'])
            with connection:
            with connection.cursor() as cursor:

        cursor.close()

        \end{lstlisting}  
        
        \textbf{docker-compose.yml}
        \begin{lstlisting}
            version: '3.9'
        services:
        mysql_206:
            env_file:
            - .env
            container_name: mysql_206
            hostname: mysql_206
            image: mysql:8
            restart: always
            command:
            - --authentication-policy=mysql_native_password
            - --character-set-server=utf8mb4
            - --collation-server=utf8mb4_unicode_ci
            - --innodb_force_recovery=0
            volumes:
            - ./mysql_206:/var/lib/mysql
            ports:
            - 3306:3306
            environment:
            
            TZ: America/Sao_Paulo
        \end{lstlisting} 
        \textbf{.env-example} 
        \begin{lstlisting}

            MYSQL_ROOT_PASSWORD = 'CHANGE-ME'
            MYSQL_DATABASE = 'CHANGE-ME'
            MYSQL_USER = 'CHANGE-ME'
            MYSQL_PASSWORD = 'CHANGE-ME'
            MYSQL_HOST = 'CHANGE-ME'
        \end{lstlisting}    

    \section{CREATE TABLE para criar tabela com PRIMARY KEY no PyMySQL}
   
    \begin{lstlisting}
        with connection:
    with connection.cursor() as cursor:
        # SQL
        cursor.execute(  # type: ignore
            'CREATE TABLE IF NOT EXISTS customers ('
            'id INT NOT NULL AUTO_INCREMENT, '
            'nome VARCHAR(50) NOT NULL, '
            'idade INT NOT NULL, '
            'PRIMARY KEY (id)'
            ') '
        )
        print(cursor)
    \end{lstlisting}

    \section{TRUNCATE e INSERT p/ limpar e criar valores na tabela com um ou mais colunas}
    \begin{lstlisting}
        import dotenv
import pymysql

TABLE_NAME = 'customers'

dotenv.load_dotenv()

connection = pymysql.connect(
    host=os.environ['MYSQL_HOST'],
    user=os.environ['MYSQL_USER'],
    password=os.environ['MYSQL_PASSWORD'],
    database=os.environ['MYSQL_DATABASE'],
    charset='utf8mb4'
)

with connection:
    with connection.cursor() as cursor:
        cursor.execute(  # type: ignore
            'CREATE TABLE IF NOT EXISTS customers ('
            f'CREATE TABLE IF NOT EXISTS {TABLE_NAME} ('
            'id INT NOT NULL AUTO_INCREMENT, '
            'nome VARCHAR(50) NOT NULL, '
            'idade INT NOT NULL, '
            'PRIMARY KEY (id)'
            ') '
        )
        print(cursor)
        # CUIDADO: ISSO LIMPA A TABELA
        cursor.execute(f'TRUNCATE TABLE {TABLE_NAME}')  # type: ignore
    connection.commit()

    # Começo a manipular dados a partir daqui

    with connection.cursor() as cursor:
        cursor.execute(  # type: ignore
            f'INSERT INTO {TABLE_NAME} '
            '(nome, idade) VALUES ("Luiz", 25) '
        )
        cursor.execute(  # type: ignore
            f'INSERT INTO {TABLE_NAME} '
            '(nome, idade) VALUES ("Luiz", 25) '
        )
        result = cursor.execute(  # type: ignore
            f'INSERT INTO {TABLE_NAME} '
            '(nome, idade) VALUES ("Luiz", 25) '
        )
        print(result)
    connection.commit()
    \end{lstlisting}

    \section{Evite SQL Injection ao usar placeholders para enviar valores }

    \begin{lstlisting}
        # Começo a manipular dados a partir daqui

    with connection.cursor() as cursor:
        cursor.execute(  # type: ignore
            f'INSERT INTO {TABLE_NAME} '
            '(nome, idade) VALUES ("Luiz", 25) '
        )
        cursor.execute(  # type: ignore
            f'INSERT INTO {TABLE_NAME} '
            '(nome, idade) VALUES ("Luiz", 25) '
        )
        result = cursor.execute(  # type: ignore
        sql = (
            f'INSERT INTO {TABLE_NAME} '
            '(nome, idade) VALUES ("Luiz", 25) '
            '(nome, idade) '
            'VALUES '
            '(%s, %s) '
        )
        data = ('Luiz', 18)
        result = cursor.execute(sql, data)  # type: ignore
        print(sql, data)
        print(result)
    connection.commit()
        
    \end{lstlisting}

    \section{Inserindo valores usando dicionários ao invés de iteráveis}
    \begin{lstlisting}
        )
        data = ('Luiz', 18)
        result = cursor.execute(sql, data)  # type: ignore
        print(sql, data)
        # print(sql, data)
        # print(result)
    connection.commit()

    with connection.cursor() as cursor:
        sql = (
            f'INSERT INTO {TABLE_NAME} '
            '(nome, idade) '
            'VALUES '
            '(%(name)s, %(age)s) '
        )
        data2 = {
            "age": 37,
            "name": "Le",
        }
        result = cursor.execute(sql, data2)  # type: ignore
        print(sql)
        print(data2)
        print(result)
    connection.commit()
    \end{lstlisting}

    \section{Lendo valores com SELECT, cursor.execute e cursor.fetchall no PyMySQL}
    
    \begin{lstlisting}
        # PyMySQL - um cliente MySQL feito em Python Puro
# Doc: https://pymysql.readthedocs.io/en/latest/
# Pypy: https://pypi.org/project/pymysql/
# GitHub: https://github.com/PyMySQL/PyMySQL
import os

import dotenv
import pymysql

TABLE_NAME = 'customers'

dotenv.load_dotenv()

connection = pymysql.connect(
    host=os.environ['MYSQL_HOST'],
    user=os.environ['MYSQL_USER'],
    password=os.environ['MYSQL_PASSWORD'],
    database=os.environ['MYSQL_DATABASE'],
    charset='utf8mb4'
)

with connection:
    with connection.cursor() as cursor:
        cursor.execute(  # type: ignore
            f'CREATE TABLE IF NOT EXISTS {TABLE_NAME} ('
            'id INT NOT NULL AUTO_INCREMENT, '
            'nome VARCHAR(50) NOT NULL, '
            'idade INT NOT NULL, '
            'PRIMARY KEY (id)'
            ') '
        )
        # CUIDADO: ISSO LIMPA A TABELA
        cursor.execute(f'TRUNCATE TABLE {TABLE_NAME}')  # type: ignore
    connection.commit()

    # Começo a manipular dados a partir daqui

    # Inserindo um valor usando placeholder e um iterável
    with connection.cursor() as cursor:
        sql = (
            f'INSERT INTO {TABLE_NAME} '
            '(nome, idade) '
            'VALUES '
            '(%s, %s) '
        )
        data = ('Luiz', 18)
        result = cursor.execute(sql, data)  # type: ignore
        # print(sql, data)
        # print(result)
    connection.commit()

    # Inserindo um valor usando placeholder e um dicionário
    with connection.cursor() as cursor:
        sql = (
            f'INSERT INTO {TABLE_NAME} '
            '(nome, idade) '
            'VALUES '
            '(%(name)s, %(age)s) '
        )
        data2 = {
            "age": 37,
            "name": "Le",
        }
        result = cursor.execute(sql, data2)  # type: ignore
        # print(sql)
        # print(data2)
        # print(result)
    connection.commit()

    # Inserindo vários valores usando placeholder e um tupla de dicionários
    with connection.cursor() as cursor:
        sql = (
            f'INSERT INTO {TABLE_NAME} '
            '(nome, idade) '
            'VALUES '
            '(%(name)s, %(age)s) '
        )
        data3 = (
            {"name": "Sah", "age": 33, },
            {"name": "Julia", "age": 74, },
            {"name": "Rose", "age": 53, },
        )
        result = cursor.executemany(sql, data3)  # type: ignore
        # print(sql)
        # print(data3)
        # print(result)
    connection.commit()

    # Inserindo vários valores usando placeholder e um tupla de tuplas
    with connection.cursor() as cursor:
        sql = (
            f'INSERT INTO {TABLE_NAME} '
            '(nome, idade) '
            'VALUES '
            '(%s, %s) '
        )
        data4 = (
            ("Siri", 22, ),
            ("Helena", 15, ),
        )
        result = cursor.executemany(sql, data4)  # type: ignore
        # print(sql)
        # print(data4)
        # print(result)
    connection.commit()

    # Lendo os valores com SELECT
    with connection.cursor() as cursor:
        sql = (
            f'SELECT * FROM {TABLE_NAME} '
        )
        cursor.execute(sql)  # type: ignore
        data5 = cursor.fetchall()  # type: ignore

        for row in data5:
            print(row)
    \end{lstlisting}
    
    \section{Lendo valores com WHERE (mais uma vez, explico cuidados com SQL) }
   
    \begin{lstlisting}
        data4 = (
            ("Siri", 22, ),
            ("Helena", 15, ),
            ("Luiz", 18, ),
        )
        result = cursor.executemany(sql, data4)  # type: ignore
        # print(sql)

    # Lendo os valores com SELECT
    with connection.cursor() as cursor:
        menor_id = int(input('Digite o menor id: '))
        maior_id = int(input('Digite o maior id: '))

        sql = (
            f'SELECT * FROM {TABLE_NAME} '
            'WHERE id BETWEEN %s AND %s  '
        )
        cursor.execute(sql)  # type: ignore

        cursor.execute(sql, (menor_id, maior_id))  # type: ignore
        print(cursor.mogrify(sql, (menor_id, maior_id)))  # type: ignore
        data5 = cursor.fetchall()  # type: ignore

        for row in data5:

    \end{lstlisting}


    \section{Apagando valores com DELETE, WHERE e placeholders no PyMySQL}
    
    \begin{lstlisting}
        # Lendo os valores com SELECT
        with connection.cursor() as cursor:
            menor_id = int(input('Digite o menor id: '))
            maior_id = int(input('Digite o maior id: '))
            # menor_id = int(input('Digite o menor id: '))
            # maior_id = int(input('Digite o maior id: '))
            menor_id = 2
            maior_id = 4
    
            sql = (
                f'SELECT * FROM {TABLE_NAME} '
                'WHERE id BETWEEN %s AND %s  '
            )
    
            cursor.execute(sql, (menor_id, maior_id))  # type: ignore
            print(cursor.mogrify(sql, (menor_id, maior_id)))  # type: ignore
            # print(cursor.mogrify(sql, (menor_id, maior_id)))  # type: ignore
            data5 = cursor.fetchall()  # type: ignore
    
            for row in data5:
            # for row in data5:
            # print(row)
    
        # Apagando com DELETE, WHERE e placeholders no PyMySQL
        with connection.cursor() as cursor:
            sql = (
                f'DELETE FROM {TABLE_NAME} '
                'WHERE id = %s'
            )
            print(cursor.execute(sql, (1,)))  # type: ignore
            connection.commit()
    
            cursor.execute(f'SELECT * FROM {TABLE_NAME} ')  # type: ignore
    
            for row in cursor.fetchall():  # type: ignore
                print(row)
    \end{lstlisting}

    \section{Editando com UPDATE, WHERE e placeholders no PyMySQL}
    
    \begin{lstlisting}
        f'DELETE FROM {TABLE_NAME} '
        'WHERE id = %s'
    )
    print(cursor.execute(sql, (1,)))  # type: ignore
    cursor.execute(sql, (1,))  # type: ignore
    connection.commit()

    cursor.execute(f'SELECT * FROM {TABLE_NAME} ')  # type: ignore

    # for row in cursor.fetchall():  # type: ignore
    #     print(row)

# Editando com UPDATE, WHERE e placeholders no PyMySQL
with connection.cursor() as cursor:
    sql = (
        f'UPDATE {TABLE_NAME} '
        'SET nome=%s, idade=%s '
        'WHERE id=%s'
    )
    cursor.execute(sql, ('Eleonor', 102, 4))  # type: ignore
    cursor.execute(f'SELECT * FROM {TABLE_NAME} ')  # type: ignore

    for row in cursor.fetchall():  # type: ignore
        print(row)
connection.commit()
    \end{lstlisting}

    \section{Trocando o cursor para retornar dicionários - pymysql.cursors.DictCursor}
  
    \begin{lstlisting}
        
import dotenv
import pymysql
import pymysql.cursors

TABLE_NAME = 'customers'

@@ -16,7 +17,8 @@
    user=os.environ['MYSQL_USER'],
    password=os.environ['MYSQL_PASSWORD'],
    database=os.environ['MYSQL_DATABASE'],
    charset='utf8mb4'
    charset='utf8mb4',
    cursorclass=pymysql.cursors.DictCursor,
)

with connection:
@@ -148,6 +150,10 @@
        cursor.execute(sql, ('Eleonor', 102, 4))  # type: ignore
        cursor.execute(f'SELECT * FROM {TABLE_NAME} ')  # type: ignore

        # for row in cursor.fetchall():  # type: ignore
        #     _id, name, age = row
        #     print(_id, name, age)

        for row in cursor.fetchall():  # type: ignore
            print(row)
    connection.commit()
    \end{lstlisting}

    \section{rowcount, rownumber e lastrowid para detalhes de consultas executadas}
    
    \begin{lstlisting}
        import pymysql.cursors

TABLE_NAME = 'customers'
CURRENT_CURSOR = pymysql.cursors.SSDictCursor
CURRENT_CURSOR = pymysql.cursors.DictCursor

dotenv.load_dotenv()

@@ -152,18 +152,26 @@
            'WHERE id=%s'
        )
        cursor.execute(sql, ('Eleonor', 102, 4))
        cursor.execute(f'SELECT * FROM {TABLE_NAME} ')

        print('For 1: ')
        for row in cursor.fetchall_unbuffered():
            print(row)
        cursor.execute(
            f'SELECT id from {TABLE_NAME} ORDER BY id DESC LIMIT 1'
        )
        lastIdFromSelect = cursor.fetchone()

        resultFromSelect = cursor.execute(f'SELECT * FROM {TABLE_NAME} ')

            if row['id'] >= 5:
                break
        data6 = cursor.fetchall()

        print()
        print('For 2: ')
        # cursor.scroll(-1)
        for row in cursor.fetchall_unbuffered():
        for row in data6:
            print(row)

        print('resultFromSelect', resultFromSelect)
        print('len(data6)', len(data6))
        print('rowcount', cursor.rowcount)
        print('lastrowid', cursor.lastrowid)
        print('lastrowid na mão', lastIdFromSelect)

        cursor.scroll(0, 'absolute')
        print('rownumber', cursor.rownumber)

    connection.commit()
    \end{lstlisting}

    \section{SSCursor, SSDictCursor e scroll para conjuntos de dados muito grandes}
  
    \begin{lstlisting}
        # Pypy: https://pypi.org/project/pymysql/
# GitHub: https://github.com/PyMySQL/PyMySQL
import os
from typing import cast

import dotenv
import pymysql
import pymysql.cursors

TABLE_NAME = 'customers'
CURRENT_CURSOR = pymysql.cursors.SSDictCursor

dotenv.load_dotenv()

@@ -18,7 +20,7 @@
    password=os.environ['MYSQL_PASSWORD'],
    database=os.environ['MYSQL_DATABASE'],
    charset='utf8mb4',
    cursorclass=pymysql.cursors.DictCursor,
    cursorclass=CURRENT_CURSOR,
)

with connection:
@@ -142,18 +144,26 @@

    # Editando com UPDATE, WHERE e placeholders no PyMySQL
    with connection.cursor() as cursor:
        cursor = cast(CURRENT_CURSOR, cursor)

        sql = (
            f'UPDATE {TABLE_NAME} '
            'SET nome=%s, idade=%s '
            'WHERE id=%s'
        )
        cursor.execute(sql, ('Eleonor', 102, 4))  # type: ignore
        cursor.execute(f'SELECT * FROM {TABLE_NAME} ')  # type: ignore
        cursor.execute(sql, ('Eleonor', 102, 4))
        cursor.execute(f'SELECT * FROM {TABLE_NAME} ')

        # for row in cursor.fetchall():  # type: ignore
        #     _id, name, age = row
        #     print(_id, name, age)
        print('For 1: ')
        for row in cursor.fetchall_unbuffered():
            print(row)

            if row['id'] >= 5:
                break

        for row in cursor.fetchall():  # type: ignore
        print()
        print('For 2: ')
        # cursor.scroll(-1)
        for row in cursor.fetchall_unbuffered():
            print(row)
    connection.commit()
    \end{lstlisting}
        


    \end{document}
    \section{}
    \textbf{}
    \begin{lstlisting}
    \end{lstlisting}

