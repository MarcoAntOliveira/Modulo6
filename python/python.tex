
    \documentclass{article}
    \usepackage[legalpaper, left=1 cm, right=1cm, top=0.5cm, bottom=0.5cm] {geometry}
    \date{} % Remove a exibição da data
    \usepackage{xcolor}
    \usepackage{listings}
    \usepackage{graphicx}
    \usepackage{hyperref} % Para criar links
    \usepackage[utf8]{inputenc}
    \usepackage[T1]{fontenc}

    \lstset{
    language=python,
    basicstyle=\ttfamily,
    keywordstyle=\bfseries\color{blue},
    commentstyle=\color{blue},
    stringstyle=\color{red!70!black},
    numberstyle=\tiny,
    stepnumber=1,
    numbersep=5pt,
    backgroundcolor=\color{white},
    breaklines=true,
    breakautoindent=true,
    showspaces=false,
    showstringspaces=false,
    showtabs=false,
    tabsize=2,
    literate={~}{{\textasciitilde}}1, % Trata ~ como um caractere normal
    extendedchars=true, % Permite caracteres estendidos (acentos, etc.)
    inputencoding=utf8, % Define a codificação de entrada como UTF-8
    literate={á}{{\'a}}1 {ã}{{\~a}}1 {ç}{{\c{c}}}1
    }
    \title{Modulo 6}
    \begin{document}
    \maketitle
    \tableofcontents
    \section{Criando meu primeiro arquivo do SQLite (db.sqlite3)}
    \begin{lstlisting}
        import sqlite3
from pathlib import Path

ROOT_DIR = Path(__file__).parent
DB_NAME = 'db.sqlite3'
DB_FILE = ROOT_DIR / DB_NAME

connection = sqlite3.connect(DB_FILE)
cursor = connection.cursor()

# SQL

cursor.close()
connection.close()
    \end{lstlisting}

    \section{Criando minha primeira tabela no SQLite3 (DBeaver opcional)}
    \begin{lstlisting}
    ROOT_DIR = Path(__file__).parent
DB_NAME = 'db.sqlite3'
DB_FILE = ROOT_DIR / DB_NAME
TABLE_NAME = 'customers'

connection = sqlite3.connect(DB_FILE)
cursor = connection.cursor()

# SQL
cursor.execute(
    f'CREATE TABLE IF NOT EXISTS {TABLE_NAME}'
    '('
    'id INTEGER PRIMARY KEY AUTOINCREMENT,'
    'name TEXT,'
    'weight REAL'
    ')'
)
connection.commit()

cursor.close()
connection.close()
    \end{lstlisting}

    \section{inserindo valores (INSERT INTO), DELETE sem WHERE e zerando a sqlite}
    \begin{lstlisting}
        connection = sqlite3.connect(DB_FILE)
cursor = connection.cursor()

# CUIDADO: fazendo delete sem where
cursor.execute(
    f'DELETE FROM {TABLE_NAME}'
)
cursor.execute(
    f'DELETE FROM sqlite_sequence WHERE name="{TABLE_NAME}"'
)
connection.commit()

# Cria a tabela
cursor.execute(
    f'CREATE TABLE IF NOT EXISTS {TABLE_NAME}'
    '('
)
connection.commit()

# Registrar valores nas colunas da tabela
# CUIDADO: sql injection
cursor.execute(
    f'INSERT INTO {TABLE_NAME} '
    '(id, name, weight) '
    'VALUES '
    '(NULL, "Helena", 4), (NULL, "Eduardo", 10)'
)
connection.commit()

cursor.close()
connection.close()
    \end{lstlisting}

    \section{Usando placeholders}
    \textbf{Esse codigo é uma continuação do codigo anterior}
    \begin{lstlisting}
        connection.commit()

# Registrar valores nas colunas da tabela
# CUIDADO: sql injection
cursor.execute(
sql = (
    f'INSERT INTO {TABLE_NAME} '
    '(id, name, weight) '
    '(name, weight) '
    'VALUES '
    '(NULL, "Helena", 4), (NULL, "Eduardo", 10)'
    '(?, ?)'
)
cursor.execute(sql, ['Joana', 4])
connection.commit()
print(sql)

cursor.close()
connection.close()
    \end{lstlisting}

    \section{Inserindo vários valores com execute many}
    \begin{lstlisting}
        'VALUES '
        '(?, ?)'
    )
    cursor.execute(sql, ['Joana', 4])
    # cursor.execute(sql, ['Joana', 4])
    cursor.executemany(
        sql,
        (
            ('Joana', 4), ('Luiz', 5)
        )
    )
    connection.commit()
    print(sql)
    \end{lstlisting}
    \section{execute e executemany com dicionários e lista de dicionários}
    \begin{lstlisting}
        f'INSERT INTO {TABLE_NAME} '
    '(name, weight) '
    'VALUES '
    '(?, ?)'
    '(:nome, :peso)'
)
# cursor.execute(sql, ['Joana', 4])
cursor.executemany(
    sql,
    (
        ('Joana', 4), ('Luiz', 5)
    )
)
# cursor.executemany(
#     sql,
#     (
#         ('Joana', 4), ('Luiz', 5)
#     )
# )
cursor.execute(sql, {'nome': 'Sem nome', 'peso': 3})
cursor.executemany(sql, (
    {'nome': 'Joãozinho', 'peso': 3},
    {'nome': 'Maria', 'peso': 2},
    {'nome': 'Helena', 'peso': 4},
    {'nome': 'Joana', 'peso': 5},
))
connection.commit()
print(sql)


    \end{lstlisting}

    \section{SELECT do SQL com fetch no SQLite3 do Python}
    \textbf{main.py}
    \begin{lstlisting}
        {'nome': 'Joana', 'peso': 5},
        ))
        connection.commit()
        print(sql)
        
        cursor.close()
        connection.close()
        
        if __name__ == '__main__':
            print(sql)
    \end{lstlisting}

    \textbf{select.py}
    \begin{lstlisting}
        import sqlite3

from main import DB_FILE, TABLE_NAME

connection = sqlite3.connect(DB_FILE)
cursor = connection.cursor()

cursor.execute(
    f'SELECT * FROM {TABLE_NAME}'
)

for row in cursor.fetchall():
    _id, name, weight = row
    print(_id, name, weight)


print()

cursor.execute(
    f'SELECT * FROM {TABLE_NAME} '
    'WHERE id = "3"'
)
row = cursor.fetchone()
_id, name, weight = row
print(_id, name, weight)

cursor.close()
connection.close()
    \end{lstlisting}

    \section{O que é CRUD + DELETE com e sem WHERE no SQLite3 do Python}
    
    \begin{lstlisting}
        connection = sqlite3.connect(DB_FILE)
cursor = connection.cursor()

# CRUD - Create Read   Update Delete
# SQL -  INSERT SELECT UPDATE DELETE

# CUIDADO: fazendo delete sem where
cursor.execute(
    f'DELETE FROM {TABLE_NAME}'
)

# DELETE mais cuidadoso
cursor.execute(
    f'DELETE FROM sqlite_sequence WHERE name="{TABLE_NAME}"'
)
@@ -51,7 +56,6 @@
    {'nome': 'Joana', 'peso': 5},
))
connection.commit()

cursor.close()
connection.close()

    \end{lstlisting}
    
    \section{DELETE no SQLite do Python}
    \begin{lstlisting}
        {'nome': 'Joana', 'peso': 5},
        ))
        connection.commit()
        cursor.close()
        connection.close()
        
        
        if __name__ == '__main__':
            print(sql)
        
            cursor.execute(
                f'DELETE FROM {TABLE_NAME} '
                'WHERE id = "3"'
            )
            cursor.execute(
                f'DELETE FROM {TABLE_NAME} '
                'WHERE id = 1'
            )
            connection.commit()
        
            cursor.execute(
                f'SELECT * FROM {TABLE_NAME}'
            )
        
            for row in cursor.fetchall():
                _id, name, weight = row
                print(_id, name, weight)
        
            cursor.close()
            connection.close()
    \end{lstlisting}

    \section{UPDATE no SQLite com Python}
    \begin{lstlisting}
        )
    connection.commit()

    cursor.execute(
        f'UPDATE {TABLE_NAME} '
        'SET name="QUALQUER", weight=67.89 '
        'WHERE id = 2'
    )
    connection.commit()

    cursor.execute(
        f'SELECT * FROM {TABLE_NAME}'
    )

    \end{lstlisting}



    \end{document}
    \section{}
    \textbf{}
    \begin{lstlisting}
    \end{lstlisting}

